\documentclass[a4paper,12pt]{article}

\usepackage{microtype}

\title{External Threats, Linguistic Homogeneity, and Political Unification}
\author{Chen Zeng}

\begin{document}
	
	\maketitle
	
	Today we start by looking at one particular place in Florence, the Basilica di Santa Croce, who didn't receive much attention as the main Florentine cathedral, whose dome dominates the city's skyline. Nevertheless, in that place lies the making of Italy as a modern nation. It buries iconic figures, such as Michelangelo, Galileo Galilei, and Niccolò Machiavelli. Machiavelli, quite often regarded as father of modern political studies, was a fervent advocate of the unification of Italy, which didn't happen until the \textit{Risorgimento} in the 19th century. His reasons for the cause, which may seem obscure today, included the Florentine lifestyle, Latin and local vernaculars, women, god, and heroes. At the end of his most famous book---\textit{Il Principe}---he called for the liberation of ``Italy from the barbarians.''
	
	In a broader sense, to what do we unify various polities? One possible and perhaps most frequently used answer is, a nation. The English word derives from the Latin word \textit{nātiōnem}, which implies that it's related to a person's birthplace; while the Chinese word originally contains meanings of territory and warfare. Today we look at two key elements in the political unification---external security threats and linguistic homogeneity.
	
	\section{Conceptualization and Operationalization}
	
	
	
\end{document}